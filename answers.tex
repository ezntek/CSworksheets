\documentclass[./main.tex]{subfiles}
\graphicspath{{\subfix{./images/}}}

\begin{document}

\textbf{NOTE: Please mark logically. These are general solutions; solutions can take many forms. Do not be worried if your exact output format looks dissimilar from these answers. Allow slight variations in formatting, identifiers, used variables, indentation, etc.}

\subsection*{Exercise 2.1.1}
From: \ref{ex:2_1_1}

\textbf{Q1}
\begin{minted}{text}
FOR Counter <- 1 TO 5
    OUTPUT "Name"
NEXT Counter
\end{minted}

\textbf{Q2}
\begin{minted}{text}
FOR Counter <- 3 TO 7
    OUTPUT Counter
NEXT Counter
\end{minted}

\textbf{Q3}
\begin{minted}{text}
FOR Counter <- 2 TO 10 STEP 2
    OUTPUT Counter
NEXT Counter
\end{minted}

\textbf{Q3}
\begin{minted}{text}
FOR Counter <- 1 TO 12
    OUTPUT 5 * Counter
NEXT Counter
\end{minted}

\textbf{Q5}
\begin{minted}{text}
FOR Counter <- 10 TO 1 STEP -1
    OUTPUT Counter
NEXT Counter
\end{minted}

\newpage
\subsection*{Exercise 2.2.1}
From \ref{ex:2_2_1}

\textbf{Q1}
\begin{minted}{text}
FOR Begin <- 1 TO 4 // Question specifies 4-5.
    FOR Counter <- Begin TO 5
        OUTPUT Counter
    NEXT Counter
NEXT Begin
\end{minted}

\textbf{Q2}
\begin{minted}{text}
FOR Base <- 1 TO 12
    OUTPUT "Times tables for ", Base
    FOR Multiplier <- 1 TO 12
        OUTPUT Base * Multiplier
    NEXT Multiplier
NEXT Base
\end{minted}

\subsection*{Exercise 2.3.1}
From \ref{ex:2_3_1}

\textbf{NOTE: For these questions, allow both exclusive and inclusive ranges.}

\textbf{Q1}
\begin{minted}{text}
DECLARE Counter:INTEGER
Counter <- 1

WHILE Counter <= 10 DO
    OUTPUT Counter
    Counter <- Counter + 1
ENDWHILE
\end{minted}

\textbf{Q2}
\begin{minted}{text}
DECLARE Counter:INTEGER
Counter <- 20

WHILE Counter >= 0 DO    
    OUTPUT Counter
    Counter <- Counter - 2
ENDWHILE
\end{minted}

\textbf{Q3}
\begin{minted}{text}
DECLARE Name:STRING
Name <- "" // optional

WHILE Name <> "exit" DO   
    OUTPUT "what is your name"
    INPUT Name

    IF Name <> "exit"
      THEN
        OUTPUT "Hello, ", Name
    ENDIF
ENDWHILE
\end{minted}

\textbf{Q4}

Accept code that uses the variable and doesn't. The logic and condition must be identical. Allow {\ccmono BREAK} statements despite their nonexistence in the specification of pseudocode as logic is the same.

\begin{minted}{text}
DECLARE Num:INTEGER
DECLARE Valid:BOOLEAN

Valid <- TRUE
Num <- 0 // optional

WHILE Valid DO
    OUTPUT "enter a number"
    INPUT Num

    IF Num > 5 AND Num < 20
      THEN
        OUTPUT "Your input is valid"
      ELSE
        Valid <- FALSE
    ENDIF
ENDWHILE
\end{minted}

\newpage
\subsection*{Exercise 2.4.1}
From \ref{ex:2_4_1}

\textbf{Q1}
\begin{minted}{text}
DECLARE Guess:STRING
CONSTANT HomeCountry <- "China"

REPEAT
    OUTPUT "Guess my home country!"
    INPUT Guess

    // the IF statement is technically optional
    IF Guess <> HomeCountry
      THEN
        OUTPUT "Not quite right..."
    ENDIF
UNTIL Guess = HomeCountry
\end{minted}

\textbf{Q2}
\begin{minted}{text}
DECLARE Temperature:INTEGER

REPEAT
    OUTPUT "Tell me the temperature in your city..."
    INPUT Temperature

    // the IF statement is technically optional
    IF Temperature < 40
      THEN
        OUTPUT "Not bad..."
    ENDIF
UNTIL Temperature > 40

OUTPUT "Wow! your city's temperature is so hot!"
\end{minted}

\subsection*{Exercise 3.2.3}
From \ref{ex:3_2_3}

\textbf{Q1}
\begin{minted}{text}
FOR Counter <- 5 TO 1 STEP -1
    OUTPUT "Do you like ", FoodItems[Counter], "?"
NEXT Counter
\end{minted}

\textbf{Q2}
\begin{minted}{text}
CONSTANT FoodsLength <- 10
DECLARE FavoriteFoods:ARRAY[1:FoodsLength] OF STRING
\end{minted}

\textbf{Q3}
\begin{minted}{text}
FavoriteFoods[1] <- FoodItems[1]
FavoriteFoods[2] <- FoodItems[2]
FavoriteFoods[3] <- FoodItems[3]
FavoriteFoods[4] <- FoodItems[4]
FavoriteFoods[5] <- FoodItems[5]
\end{minted}

\textbf{Q4}
\begin{minted}{text}
FOR Counter <- 1 TO 5
    FavoriteFoods[Counter] <- FoodItems[Counter]
NEXT Counter
\end{minted}

\textbf{Q5}

Students may loop through 1-5 and use {\ccmono Counter + 5} as the index, generating 6-10. Students may also use FoodsLength instead of 10.

\begin{minted}{text}
DECLARE Food:STRING

FOR Counter <- 6 TO 10
    OUTPUT "Enter a food: "
    INPUT Food
    FavoriteFoods[Counter] <- Food
NEXT Counter
\end{minted}

\textbf{Q6}

Students may use FoodsLength instead of 10.

\begin{minted}{text}
FOR Counter <- 1 TO 10
    OUTPUT FavoriteFoods[Counter]
NEXT Counter
\end{minted}

\newpage
\subsection*{Exercise 3.3.3}
From \ref{ex:3_3_3}

\textbf{Q1}
\begin{minted}{text}
Line 5: FOR Row <- 3 TO 1 STEP -1
Line 6: FOR Column <- 8 TO 1 STEP -1
\end{minted}

\textbf{Q2}
\begin{minted}{text}
DECLARE ExamScores:ARRAY[1:5,1:5] OF INTEGER 
\end{minted}

\textbf{Q3}
\begin{itemize}
    \item INTEGER
    \item Boundaries depend on the implementation, but accept all negative and positive non-decimals
    \item Since the data does not need to store decimals, nor text nor characters, INTEGER makes the most sense.
\end{itemize}

\textbf{Q4}
\begin{minted}{text}
DECLARE Num:INTEGER
Num <- 0 // optional

FOR Row <- 1 TO 5
    FOR Column <- 1 TO 5
        // also allow while loops.

        REPEAT
            OUTPUT "enter a number: "
            INPUT Num
            
            IF Num < 5 AND NUM > 10
              THEN
                OUTPUT "invalid input! try again..."
            ENDIF
        UNTIL Num >= 5 AND NUM <= 10
    NEXT Column
NEXT Row
\end{minted}

\newpage
\subsection*{Exercise 3.4.1}
From \ref{ex:3_4_1}

\textbf{Q1}
\begin{minted}{text}
FOR Counter <- 1 TO 5
    OUTPUT "Name: ", StudentNames[Counter]
    
    // as long as if the student attempts to output something, the answer
    // should be accepted. the objective is to test the understanding of
    // the concept of parallel arrays bind data through the index.
    //
    OUTPUT "Science: ", StudentGrades[1]
    OUTPUT "Math: ", StudentGrades[2]
    OUTPUT "English: ", StudentGrades[3]
NEXT Counter
\end{minted}   

\textbf{Q2}

If the student attempts to calculate the median/mode, give them a pat on their back for their effort and rebeliousness.

\begin{minted}{text}
DECLARE Total:INTEGER
DECLARE Average:REAL // make sure it is REAL, do not accept INTEGER
                      // this variable is also optional

FOR Counter <- 1 TO 5
    OUTPUT "Name: ", StudentNames[Counter]
    
    // A for loop is appreciated, but not required. 
    // Here, additions are wrapped to the next line.
    Total <- StudentGrades[Counter,1]
             + StudentGrades[Counter,2]
             + StudentGrades[Counter,3]
    Average <- Total / 3

    OUTPUT "Average grade is: ", Average
NEXT Counter
\end{minted}   

\textbf{Q3}
\begin{minted}{text}
DECLARE Total:INTEGER
DECLARE Average:REAL
DECLARE LowestAverage:REAL
DECLARE LowestAverageName:STRING

LowestAverage <- 0 // LowestAverage will be accessed without
                    // being initialized in the loop.
                    // The student must initialize this.

FOR Counter <- 1 TO 5
    Total <- StudentGrades[1] + StudentGrades[2] + StudentGrades[3]
    Average <- Total / 3

    IF Average < LowestAverage
      THEN
        LowestAverageName <- StudentName[Counter]
        LowestAverage <- Average
    ENDIF
NEXT Counter

OUTPUT "The lowest scorer was ", LowestAverageName
OUTPUT "They earned a ", LowestAverage, " average."
\end{minted}

\end{document}
